\subsection{Mint}
\label{sec:20Mint}

Suppose Alice wishes to be able to transfer ownership of an amount of ERC-20 tokens under zero-knowledge, so that the following become private:
	\begin{center}
		\begin{framed}
      \begin{enumerate}
        \item The value of ERC-20 tokens being transferred.
        \item The identity of the sender of the tokens (`Alice').
        \item The identity of the recipient of the tokens.
      \end{enumerate}
    \end{framed}
  \end{center}
In order to achieve this, Alice must first convert her ERC-20 tokens into a private ERC-20 commitment. We call this act of conversion \textbf{`minting'} an ERC-20 commitment.\\
\\
In this section, we outline Nightfall's protocol for minting an ERC-20 commitment, but first, an important privacy warning is given in Figure~\ref{fig:fMintWarning}.

\begin{figure}[h]
  \begin{center}
    \begin{mdframed}[backgroundcolor=verylightred]
      \noindent
      PRIVACY WARNING\\
      \\
      \textit{
      Privacy is NOT achieved during the minting stage!
      }\\
      \\
      Minting an ERC-20 commitment initially requires Alice to transfer a certain value of ERC-20 tokens to a `Shield' contract (which thereafter holds this value in escrow). This transfer reveals the Ethereum address of the sender (Alice) as well as the value. Therefore everyone will know the owner and the underlying value being represented by the initial ERC-20 commitment which is created at this `minting' stage.\\
      \\
      Only during subsequent `transfers' of the new ERC-20 commitment, will we achieve the privacy intentions of Figure~\ref{fig:fPrivacyIntentions}
    \end{mdframed}
  \end{center}
  \caption{Privacy warning: minting alone does not achieve privacy}
  \label{fig:fMintWarning}
\end{figure}

\noindent
Suppose Alice owns ERC-20 tokens of value $c$
(denominated in the ERC-20 token's currency).
Suppose Alice wishes to create a private token commitment, representing her ownership of value $c$.\\
\\
For Alice to mint a token commitment $Z_c$ representing value $c$ on the blockchain, under zero knowledge, she follows the steps in Figure~\ref{fig:fMintAlgorithm}. Note: We avoid using $Z_A$ (which was used to stress Alice's ownership in the non-fungible section), because Alice will own more than one token commitment when we explain `transfers'.
\\

\begin{figure}[htp]
	\begin{center}
		\begin{framed}
      \begin{tabular}{p{16cm}}	
        \textbf{Fungible mint algorithm} \\
        \\
        \midrule
        \textbf{Alice's steps:}\\
        \begin{enumerate}
				  \item Generate a random salt $\sigma_c$.
				  \item Compute $Z_c := h(\;c\;|\;pk^Z_A\;|\;\sigma_c\;)$, a token commitment which represents $c$.
          \item Set public inputs $x = (\;c,\;Z_c\;)$
          \item Set private inputs $\omega = (\;pk_A^Z,\;\sigma_c\;)$
          \item Select $C_{nft-mint}(\;\omega,\;x\;)$ -- the set of constraints which are satisfied if and only if:
          \begin{enumerate}
            \item $Z_c$ equals $h(\;c\;|\;pk_A^Z\;|\;\sigma_c\;)$ (Proof that the commitment $Z_c$ hides the correct value $c$)
          \end{enumerate}
          \item Generate $\pi := P(\;p_C\;,\;x,\;\omega\;)$; a proof of knowledge of satisfying arguments $(\omega, x)\;s.t.\;C(\omega, x) = 1$. Recall: $p_C$ -- the proving key for $C$ -- will be stored on Alice's computer.
           
          The pair $(\pi, x)$ is the zk-SNARK which attests to knowledge of private inputs $\omega$ without revealing them.
          \item Send $(\pi, x)$ to the Shield contract for verification.
           
          Using web3: \texttt{fTokenShield.mint(proof, inputs, vkId)}
          %remember where the count (enumi) is up to and store it in ongoingEnumCounter:
          \setcounter{ongoingEnumCounter}{\value{enumi}}
        \end{enumerate}
        \ \\
        \midrule
        \textbf{Shield contract's steps:}\\
        \begin{enumerate}
          %resume counter
          \setcounter{enumi}{\value{ongoingEnumCounter}}
          \item Verify the proof as correct: call a Verifier contract to verify the \texttt{(proof, inputs)} pair against the verification key represented by \texttt{vkId}.
          \setcounter{ongoingEnumCounter}{\value{enumi}}
        \end{enumerate}
        \ \\
        \midrule
        \textbf{Verifier contract's steps:}\\
        \begin{enumerate}
          \setcounter{enumi}{\value{ongoingEnumCounter}}
          \item Compute \texttt{result = verify(proof, inputs, vkId)}.
          
          I.e. Verify the \texttt{(proof, inputs)} pair against the verification key.
          \item Return \texttt{result}$\in$\texttt{\{false, true\}} to the Shield contract.
          \setcounter{ongoingEnumCounter}{\value{enumi}}
        \end{enumerate}
        \ \\
        \midrule
        \textbf{Shield contract's steps:}\\
        \begin{enumerate}
          \setcounter{enumi}{\value{ongoingEnumCounter}}
          \item If \texttt{result = false}, revert.
          \item Else:
          \begin{enumerate}
            \item Transfer a value of $c$, on behalf of Alice, to the Shield Contract. I.e. store $c$ in escrow.
            \item Add $Z_c$ to the next empty leaf of the Merkle Tree.
            \item Recalculate the path to the root of the Merkle Tree from $Z_c$ for future users.
          \end{enumerate}
          \setcounter{ongoingEnumCounter}{\value{enumi}}
        \end{enumerate}
        \ \\
        \midrule
        \textbf{Alice's steps:}\\
        \begin{enumerate}
          \setcounter{enumi}{\value{ongoingEnumCounter}}
          \item Store relevant data in local database, including the leafindex of $Z_c$.
          \setcounter{ongoingEnumCounter}{0} %reset for next figure
        \end{enumerate} 
			\end{tabular}
		\end{framed}
	\end{center}
\caption{Fungible Mint Algorithm}
\label{fig:fMintAlgorithm}
\end{figure}

\newpage
\subsubsection{Details}
\label{sec:20MintDetails}

We refer to the numbered steps of Figure~\ref{fig:fMintAlgorithm}.\\
\\
\textbf{Step $1$}
\ \\
This is handled within the \hyperref[sec:ui]{UI} microservice (or within the api-gateway).\\
\\

\textbf{Steps $2 - 4$}
\ \\
These steps are handled within \hyperref[sec:f-token-controller]{\texttt{f-token-controller.js}}.\\
\\

\textbf{Steps $5 - 6$}
\ \\
These steps are handled within a \hyperref[sec:zokrates]{ZoKrates} container.\\
\\

\textbf{Step $7$}
\ \\
This transaction is handled within \hyperref[sec:f-token-zkp]{\texttt{f-token-zkp.js}}.\\
\\

\textbf{Steps $8 - 10$}
\ \\
The Verifier contract already has stored within it the object $vk_C$ (see \hyperref[sec:trustedSetup]{Trusted Setup}). It runs a verification function $V(vk_C, \pi , x)$.
\begin{align*}
  V: (vk_C, \pi_{C,x,\omega}, x) \to \{0,1\}
\end{align*}
where:
\[
    V=
\begin{cases}
    1,& \text{if } \pi_{C,x,\omega} \text{ and } x \text{ satisfy } vk_C\\
    0,& \text{otherwise}
\end{cases}
\]
\ \\



\textbf{Steps $11 - 12$}
\ \\
If the Verifier contract returns $1$ (\texttt{true}) (verified) to the Shield contract, then the Shield contract will be satisfied with Alice's commitment, and will update its persistent states:\\
\\
Suppose the Shield contract stores an ever-increasing array, $\bm{Z}$, of all token commitments which have ever been submitted by anyone.\\
\\
Suppose, prior to Alice's mint, there are $n-1$ tokens in the tree:
\begin{align*}
  \bm{Z}_{n-1} = (Z_0, Z_1, ..., Z_{n-1})
\end{align*}
The information held within $\bm{Z}_{n-1}$ may be represented by the root hash $\roott_{n-1}$ of a Merkle Tree $M_{n-1}$:\\


\begin{align*}
  \begin{forest}
    [{$\roott_{n-1}:= h\br*{
                      h\br*{
                        h\br*{
                          h\br*{
                            Z_0,Z_1
                          },
                          ...
                        },
                        h\br*{
                          h\br*{
                            Z_{n-1},0
                          },
                          0
                        }
                      },
                      0
                    }
                  $}
      [{$ h\br*{
            h\br*{
              h\br*{
                Z_0,Z_1
              },
              ...
            },
            h\br*{
              h\br*{
                Z_{n-1},0
              },
              0
            }
          }
        $}
        [{$ h\br*{
              h\br*{
                Z_0,Z_1
              },
              ...
            }
          $}
          [{$ h\br*{
                Z_0,Z_1
              }
            $}
            [{$Z_0$}][{$Z_1$}]
          ]
          [...
            [...][...]
          ]
        ]
        [{$ h\br*{
              h\br*{
                Z_{n-1},0
              },
              0
            }
          $}
          [{$ h\br*{
                Z_{n-1},0
              }
            $}
            [{$Z_{n-1}$}][0]
          ]
          [0
            [0][0]
          ]
        ]
      ]
      [0
        [0
          [0
            [0][0]
          ]
          [0
            [0][0]
          ]
        ]
        [0
          [0
            [0][0]
          ]
          [0
            [0][0]
          ]
        ]
      ]
    ]
  \end{forest}
\end{align*}

\noindent
Now that the Shield contract has been given verification that Alice's commitment, $Z_c$, does indeed hide the value $c$, the Shield contract will do the following:\\

\begin{itemize}
  \item Append the commitment $Z_c$ to the ever-increasing array of tokens, $\bm{Z}_{n-1}$, so that $\bm{Z}_{n}=(Z_0, Z_1, ... Z_{n-1}, Z_c)$
  \item Recalculate a Merkle Root $\roott_{n}$ of $M_{n}$:
  \\
  \begin{align*}
    \begin{forest}
      [{$\roott_{n}:= h\br*{
                      h\br*{
                        h\br*{
                          h\br*{
                            Z_0,Z_1
                          },
                          ...
                        },
                        h\br*{
                          h\br*{
                            Z_{n-1},Z_c
                          },
                          0
                        }
                      },
                      0
                    }
                  $}
        [{$ h\br*{
              h\br*{
                h\br*{
                  Z_0,Z_1
                },
                ...
              },
              h\br*{
                h\br*{
                  Z_{n-1},Z_c
                },
                0
              }
            }
          $}
          [{$ h\br*{
                h\br*{
                  Z_0,Z_1
                },
                ...
              }
            $}
            [{$ h\br*{
                  Z_0,Z_1
                }
              $}
              [{$Z_0$}][{$Z_1$}]
            ]
            [...
              [...][...]
            ]
          ]
          [{$ h\br*{
                h\br*{
                  Z_{n-1},Z_c
                },
                0
              }
            $}
            [{$ h\br*{
                  Z_{n-1},Z_c
                }
              $}
              [{$Z_{n-1}$}][{$Z_c$}]
            ]
            [0
              [0][0]
            ]
          ]
        ]
        [0
          [0
            [0
              [0][0]
            ]
            [0
              [0][0]
            ]
          ]
          [0
            [0
              [0][0]
            ]
            [0
              [0][0]
            ]
          ]
        ]
      ]
    \end{forest}
  \end{align*}

  \item Append $\roott_{n}$ to an ever-increasing array $\rootsList=(\roott_0, \roott_1,..., \roott_{n-1}, \roott_{n})$
\end{itemize}


\textbf{Step $13$}
\ \\
Alice will store all important information in her private database.\\


